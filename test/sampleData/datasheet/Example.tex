\documentclass[a4paper,12pt,oneside,pdflatex,italian,final,twocolumn]{article}

\usepackage[utf8]{inputenc}
\usepackage{parallel}
\usepackage{siunitx}
\usepackage{booktabs}
\usepackage{fancyhdr}

\usepackage[export]{adjustbox}
\usepackage[margin=0.5in]{geometry}
\addtolength{\topmargin}{0in}

\usepackage{libertine}
\renewcommand*\familydefault{\sfdefault}  %% Only if the base font of the document is to be sans serif
\usepackage[T1]{fontenc}

\title{ Example }
\author{ Nick Felker }
\date{ 2019 }

\begin{document}

\pagestyle{fancy}

\lhead{ Nick Felker }
\chead { 2019 }
\rhead{ Example v0.1.0 }


\onecolumn


\begin{figure}
\begin{minipage}{0.47\textwidth}

\section{Overview}
    Example of a package
    \begin{itemize}
        \item Device addresses:
          16,
          32,
          48
        \item Address type 7-bit
    \end{itemize}


\end{minipage}
\hfill

\end{figure}


\section{Register Description}
\begin{itemize}
\item First example - An 8-bit register
\item Second example - A 16-bit register
\item Third example - A 32-bit register
\end{itemize}

\section{Technical specification}
\centering
\begin{tabular}{lcrr}
\toprule
 & Register Name & Register Address & Register Length \\
\midrule
RegisterA & First example & 0 & 8 \\
RegisterB & Second example & 1 & 16 \\
RegisterC & Third example & 2 & 32 \\
\bottomrule
\end{tabular}

\raggedright

\section{Fields}



\raggedright

\subsection{Register RegisterA}
\centering
\begin{tabular}{lcr}
\toprule
  & Field Name & Bits \\
\midrule
FieldA & Read-only fields for RegisterA &
7:4
\\
FieldB & Write-only fields for RegisterA &
3:2
\\
FieldC & Read/write field for RegisterA &
1
\\
FieldD & This field should never appear &
0
\\
\bottomrule

\end{tabular}


\raggedright

\subsubsection{Field FieldB }

This is fewer bits

\begin{itemize}
\item VAL\_1 (1) - Value 1
\item VAL\_2 (2) - Value 2
\item VAL\_3 (4) - Value 3
\item VAL\_4 (8) - Value 4
\end{itemize}




\raggedright

\section{Functions}

\centering
\begin{tabular}{lc}
\toprule
  & Description \\
\midrule
Return & Computes and returns \\
\bottomrule
\end{tabular}


\raggedright
\subsection{Function Return }
Computes and returns \\

\centering
\begin{tabular}{lcr}
\toprule
  & Inputs & Return \\
\midrule
Array &
&
summation, 
summation
\\
Number &
&
summation
\\
Void &
&
void
\\
\bottomrule
\end{tabular}



\raggedright

\end{document}