\documentclass[a4paper,12pt,oneside,pdflatex,italian,final,twocolumn]{article}

\usepackage[utf8]{inputenc}
\usepackage{parallel}
\usepackage{siunitx}
\usepackage{booktabs}
\usepackage{fancyhdr}

\usepackage[export]{adjustbox}
\usepackage[margin=0.5in]{geometry}
\addtolength{\topmargin}{0in}

\usepackage{libertine}
\renewcommand*\familydefault{\sfdefault}  %% Only if the base font of the document is to be sans serif
\usepackage[T1]{fontenc}

\title{ MCP9808 }
\author{ Joe Smith }
\date{ 2019 }

\begin{document}

\pagestyle{fancy}

\lhead{ Joe Smith }
\chead { 2019 }
\rhead{ MCP9808 v0.1.0 }


\onecolumn


\begin{figure}
\begin{minipage}{0.47\textwidth}

\section{Overview}
    This is a test description
    \begin{itemize}
        \item Device address 24
        \item Address type 7-bit
    \end{itemize}


\end{minipage}
\hfill

\end{figure}


\section{Register Description}
\begin{itemize}
\item Configuration Register - The MCP9808 has a 16-bit Configuration register (CONFIG) that
allows the user to set various functions for a robust temperature
monitoring system.
Bits 10 through 0 are used to select the temperature alert output
hysteresis, device shutdown or Low-Power mode, temperature boundary
and critical temperature lock, and temperature Alert output
enable/disable.
In addition, Alert output condition (output set for TUPPER and
TLOWER temperature boundary or TCRIT only), Alert output status
and Alert output polarity and mode (Comparator Output or Interrupt
Output mode) are user-configurable.

\end{itemize}

\section{Technical specification}
\centering
\begin{tabular}{lcrr}
\toprule
 & Register Name & Register Address & Register Length \\
\midrule
configuration & Configuration Register & 1 & 16 \\
\bottomrule
\end{tabular}

\raggedright

\section{Fields}

\centering
\begin{tabular}{lcrr}
\toprule
  & Field Name & Register & Bits \\
\midrule
limitHysteresis & TUPPER and TLOWER Limit Hysteresis bits & configuration &
10:9
\\

shutdownMode & Shutdown Mode bit & configuration &
8
\\

\bottomrule
\end{tabular}

\raggedright



\end{document}